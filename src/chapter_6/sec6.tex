\documentclass[../../main.tex]{subfiles}

\begin{document}

\subsection*{6.6 Transactions in SQL}

\subsubsection*{Exercise 6.1}

a)

\begin{lstlisting}[language=sql]
  SET TRANSACTION READ ONLY
    ISOLATION LEVEL READ COMMITTED
  BEGIN TRANSACTION
  SELECT model, price
  FROM PC
  WHERE speed = providedSpeed, ram = providedRAM
  COMMIT
\end{lstlisting}

b)

\begin{lstlisting}[language=sql]
  SET TRANSACTION READ WRITE
    ISOLATION LEVEL SERIALIZABLE
  BEGIN TRANSACTION
  DELETE FROM PC
  WHERE PC.model = providedModel
  DELETE FROM Product
  WHERE Product.model = providedModel
  COMMIT
\end{lstlisting}

c)

\begin{lstlisting}[language=sql]
  SET TRANSACTION READ WRITE
    ISOLATION LEVEL SERIALIZABLE
  BEGIN TRANSACTION
  UPDATE PC
  SET price = price - 100
  WHERE PC.model = providedModel
  COMMIT
\end{lstlisting}

d)

\begin{lstlisting}[language=sql]
  SET TRANSACTION READ WRITE
    ISOLATION LEVEL SERIALIZABLE
  BEGIN TRANSACTION
  UPDATE PC AS P
  SET (maker, model, speed, ram, hd, price)
  WHERE IF(maker=P.maker AND model=P.model AND speed=P.speed AND
           ram=P.ram AND hd=P.hd AND price=P.price)
        PRINT 'Error. There is model like that.'
        COMMIT
  COMMIT
\end{lstlisting}

\subsubsection*{Exercise 6.2}

Here, I only give the answer for the first problem, it is easy to give an example for
other situations. When looking up the \verb|PC|, we may UPDATE it at the same time.

\subsubsection*{Exercise 6.3}

Here, I only give the answer for the first problem, it is easy to give an example for
other situations. When the ISOLATION LEVEL is \verb|READ UNCOMMITTED|, we may look up
the dirty data.

\subsubsection*{Exercise 6.4}

\begin{itemize}
  \item \emph{Serializable}: With a lock-based concurrency control serializability requires read and write
        locks to be released at the end of the transaction.When using non-lock based
        concurrency control, no locks are acquired; however, if the system detects a
        write collision among several concurrent transactions, only one of them is
        allowed to commit. Here the Transaction \verb|T|,has to acquire locks before reading
        and writing. So, there will be no impact of other processes running.
  \item \emph{Repeatable reads}: With a lock-based concurrency control this keeps read and write locks
        until the end of the transaction. Write skew is possible at this isolation level, a situation
        where two writes are allowed to the same column in a row by two different writers, resulting
        in the row having data that is a mix of the two transactions. Here the Transaction \verb|T|,
        has to acquire locks before reading and writing. So, there will be no impact of other
        processes running.
  \item \emph{Read committed}: A lock-based concurrency control this keeps write locks until the end
        of the transaction, but read locks are released as soon as the \verb|SELECT| operation is performed.
        Read committed is an isolation level that guarantees that any data read is committed at the
        moment it is read. It simply restricts the reader from seeing any intermediate,
        uncommitted, 'dirty' read. Here the data read by Transaction \verb|T| will always be committed so
        there should be no issue with other process.
  \item \emph{Read uncommitted}: This is the lowest isolation level. In this level, dirty reads
        are allowed, so one transaction may see not-yet-committed changes made by other transactions.
        Here it could be a situation where data updated by other transactions are not available
        and the Transaction \verb|T| keeps running for ever
\end{itemize}

\end{document}
