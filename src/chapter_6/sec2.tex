\documentclass[../../main.tex]{subfiles}

\begin{document}

\subsection*{6.2 Query Involving More Than One Relation}

\subsubsection*{Exercise 2.1}

a)

\begin{lstlisting}[language=sql]
  SELECT name
  FROM StarsIn, MovieStar
  WHERE gender = 'M' AND name = starName
        AND movieTitle = 'Titanic';
\end{lstlisting}

b)

\begin{lstlisting}[language=sql]
  SELECT starName
  FROM StarsIn, Movies
  WHERE gender = 'M' AND name = starName
        AND year = 1995
        AND studioName = 'MGM';
\end{lstlisting}

c)

\begin{lstlisting}[language=sql]

\end{lstlisting}

\subsubsection*{Exercise 2.2}

A systematic way to handle this problem is to create
a tuple variable for every $R_{i}, i = 1,2,\dots,n$,
whether we need to or not. That is, the \verb|FROM|
clause is

\begin{lstlisting}[language=sql]
  FROM R1 AS T1, R2 AS T2, ...
\end{lstlisting}

Now, build the \verb|WHERE| clause from $C$ by
replacing every reference to some attribute $A$
of $R_{i}$ by $T_{i}.A$. Also, build the
\verb|SELECT| clause from list of attributes $L$
by replacing every attribute $A$ of $R_{i}$ by
$T_{i}.A$.

\subsubsection*{Exercise 2.3}

\begin{lstlisting}[language=sql]
  SELECT L
  FROM R1 NATURAL JOIN R2, ...
  ON R1.column = R2.column AND ...
  WHERE C;
\end{lstlisting}

\end{document}
