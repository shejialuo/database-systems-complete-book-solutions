\documentclass[../../main.tex]{subfiles}

\begin{document}

\subsection{18.3 Enforcing Serializability by Locks}

\subsubsection*{Exercise 3.1}

a)

Suppose that we have the schedule derived from the given transactions.

$$
r_{1}(A); r_{2}(B); w_{2}(B); r_{2}(A); w_{1}(A); r_{1}(B); w_{1}(B); w_{2}(A)
$$

Note that $r_{1}(A)$ must hold a lock on $A$, since a transaction is granted with
lock before its read and write action. Likewise, $r_{2}(A)$ must also hold a lock
on $A$.

Since the unlock of $A$ for $r_{1}(A)$ is executed after $w_{1}(A)$ and $r_{2}(A)$
appears before $w_{1}(A)$. Therefore, the schedule is prohibited.

b)

$$
\frac{4!}{2! \times (4 - 2)!} \times \frac{4!}{2! \times (4 - 2)!} + 2 = 38
$$

c)

$$
\frac{4!}{2! \times (4 - 2)!} \times \frac{4!}{2! \times (4 - 2)!} = 36
$$


d)

The number of conflict serializable schedule is 2.

e)

All legal schedules are serializable. It is impossible for a legal schedule
to be unserializable.

\subsubsection*{Exercise 3.2}

Need community help.

\subsubsection*{Exercise 3.3}

Need community help.

\subsubsection*{Exercise 3.4}

Need community help.

\end{document}
