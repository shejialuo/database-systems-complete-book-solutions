\documentclass[../../main.tex]{subfiles}

\begin{document}

\subsection{7.1 Keys and Foreign Keys}

\subsubsection*{Exercise 1.1}

a)

\begin{lstlisting}[language=sql]
  FOREIGN KEY (producerC#) REFERENCES MovieExec(cert#);
\end{lstlisting}

b)

\begin{lstlisting}[language=sql]
  FOREIGN KEY (producerC#) REFERENCES MovieExec(cert#)
  ON UPDATE SET NULL  -- if MOvie exec cert# is changed, set to null
                      -- in practice this should not be used
                      -- since it will lose referential integrity
                      -- ideally, it should be CASCADE
  ON DELETE SET NULL; -- if Movie exec does not exist
\end{lstlisting}

c)

\begin{lstlisting}[language=sql]
  FOREIGN KEY (producerC#) REFERENCES MovieExec(cert#)
    ON UPDATE CASCADE
    ON DELETE CASCADE;
\end{lstlisting}

d)

\begin{lstlisting}[language=sql]
  FOREIGN KEY (movieTitle, movieYear)
    REFERENCES Movies(title, year);
-- by default, UPDATE and CASCADE are restrict
\end{lstlisting}

e)

\begin{lstlisting}[language=sql]
  FOREIGN KEY (starName) REFERENCES MovieStar(name)
    ON DELETE CASCADE;
\end{lstlisting}

\subsubsection*{Exercise 1.2}

No.

Assume the foreign key constraint ($a$) in relation R that references relation S.

Thus:

\begin{itemize}
\item R must contain the attribute(s) $a$
\item $a$ is a candidate key of S
\item if a tuple in R contains $a$, $a$ exists in S.
\end{itemize}

Therefore, we can't enforce that every Movie has a tuple in StarsIn because:

\begin{itemize}
\item Movies does not contain the candidate key attributes of StarsIn
\end{itemize}

\subsubsection*{Exercise 1.3}

\begin{itemize}
\item Product: model
\item PC: model
\item Laptop: model
\item Printer: model  
\end{itemize}

In PC, Laptop and Printer, we need a foreign key constraint:

\begin{lstlisting}[language=sql]
CREATE TABLE Product(
  ...
  PRIMARY KEY (model)
);
CREATE TABLE PC(
  ...
  PRIMARY KEY (model),
  FOREIGN KEY (model) REFERENCES Product(model)
);
CREATE TABLE Laptop(
  ...
  PRIMARY KEY (model),
  FOREIGN KEY (model) REFERENCES Product(model)
);
CREATE TABLE Printer(
  ...
  PRIMARY KEY (model),
  FOREIGN KEY (model) REFERENCES Product(model)
);

\end{lstlisting}

\end{document}
