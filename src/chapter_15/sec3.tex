\documentclass[../../main.tex]{subfiles}

\begin{document}

\subsection{15.3 Nested-Loop Joins}

\subsubsection*{Exercise 3.1}

Need community help.

\subsubsection*{Exercise 3.2}

According to the formula:

$$
B(S) + ((B(S)B(R))) / (M - 1)
$$

We have

$$
10000 + (10000 ^ 2) / 999 = 110101
$$

\subsubsection*{Exercise 3.3}

According to the formula:

$$
B(S) + ((B(S)B(R))) / (M - 1) = IO
$$

We can have:

$$
M = (IO - B(S)) / ((B(S)B(R))) + 1
$$

(a)

$$
M = 1112
$$

(b)

$$
M = 6.667
$$

(c)

$$
M = 20.001
$$

\subsubsection*{Exercise 3.4}

a)

A block based nested loop join, used to join two relation $R$ and $S$ in a
relational database, but nested loop join, used to two relations and the
outer and inner join respectively.

b)

case 1: $R$ is not clustered and smaller. Cost of reading all tuples of $S$
(clustered), $T(S)+B(S)$ Cost of reading $R$ tuples and the cost of join with $S$ in
the main memory is: $T(R)+B(S)B(R)/M $ The total cost is $T(R)+
B(S)B(R)/M+T(S)+T(R)$

case 2: $S$ is not clustered and smaller. Cost of reading all tuples of $S$
(clustered), $T(R)+B(R)$ Cost of reading $S$ tuples and the cost of join with $R$
in the main memory is: $T(S)+B(S)B(R)/M$ The total cost is $T(S)+
B(S)B(R)/M+T(R)+B(R)$

\subsubsection*{Exercise 3.5}

Need community help.

\end{document}
