\documentclass[../../main.tex]{subfiles}

\begin{document}

\subsection{15.7 Buffer Management}

\subsubsection*{Exercise 7.1}

a)

The terms $B(R)$ and $B(S)$ represents binary relations between the relations
$R$ and $S$. In one-pass, there is an approximate requirement for the binary
operation between the relations $R$ and $S$ as:

$$
\min(B(R), B(S)) \leq M
$$

This approximation means that the one buffer is used to read the blocks
of larger relation and for smaller relations, it requires $M$ buffers
additionally with the main memory structure. According to the one-pass
algorithm and given terms, the relation either $R$ or $S$ must fit into the
memory. So, the approximation rule would be:

$$
min(B(R), B(S)) \geq \frac{M}{2}
$$


b)

This algorithm might work properly only if the one-pass union,
intersection and difference between the relations $R_{i}$ and $S_{i}$
whose sizes are found to be $\frac{B(R)}{M - 1}$ and $\frac{B(S)}{M - 1}$
respectively as it is known, the one-pass algorithm requires
operand and it occupies at most  $M - 1$ blocks.

Therefore the two-pass hash based algorithms requires at least
$\min(B(R), B(S)) \leq M^{2}$ approximately.

Since the worst case of two-pass algorithm contains $\frac{M}{2}$
blocks. The memory between $M$ and $\frac{M}{2}$ requires approximately

$$
\min(B(R), B(S)) \leq \frac{M^2}{4}
$$

c)

$$
\max(B(R), B(S)) \leq \frac{M^2}{4}
$$

\subsubsection*{Exercise 7.2}

a)

Following the concept of FIFO the new blocks occupies the buffer by
emptying the current longest block in buffer. When doing the nested loop
join operation, the FIFO pointer pointing the longest block of the buffer
and it need not points out the first tuple in the buffer.

So it will not improve the number of disk I/O's on nested loop
join operations.

b)

The clock algorithm contains the handle which places the disk on the
available buffer by rotating the handle in clockwise direction. While doing
the nested loop join, approach also doesn't searches for the first tuple as
the handle of the clock is present on the available space of the buffer.

It will not improve the number of disk I/O's on nested loop join
operations.

\subsubsection*{Exercise 7.3}

Need community help.

\end{document}

