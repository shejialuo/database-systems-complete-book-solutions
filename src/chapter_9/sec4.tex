\documentclass[../../main.tex]{subfiles}

\begin{document}

\subsection{9.4 Stored Procedures}

\subsection*{Exercise 4.1}

a)

\begin{lstlisting}[language=sql]
  CREATE FUNCTION Movie (n CHAR(20), w INT) RETURN INT
    SELECT name, netWorth
    FROM Studio, MovieExec
    WHERE name=n AND netWorth=w;
\end{lstlisting}

b)

\begin{lstlisting}[language=sql]
  CREATE FUNCTION nameAddress(n CHAR(20), a CHAR(20)) RETURN BOOLEAN
    IF(SELECT name.S, name.E
       FROM MovieStar.S, MovieExec.E
       WHERE n=name.S AND NOT (n=name.E)) RETURN 1;
    ELSEIF 2 <= (SELECT name.S,name.E
                 FROM MovieStar.S, MovieExec.E
                 WHERE n=name.E AND NOT (n=name.S) )
    ELSEIF 3 <= (SELECT name.S, name.E
                 FROM MovieStar.S, MovieExec.E
                 WHERE n=name.E AND n=name.S)
    ELSEIF 4 <= (SELECT name.S, name.E
                 FROM MovieStar.S, MovieExec.E
                 WHERE NOT(n=name.E) AND NOT(n=name.S))
    END IF;
\end{lstlisting}

c)

\begin{lstlisting}[language=sql]
  CREATE FUNCTION sName (s CHAR(20)) RETURN CHAR
    IF(SELECT *
       FROM Movies
       WHERE studioName=s
       GROUP BY length
       HAVING COUNT(*)=2)
    THEN RETURN sName;
    ELSE IF (SELECT *
             FROM Movies
             WHERE studioName=s
             GROUP BY length
             HAVING COUNT(*)=1)
         THEN RETURN 'There is no second longest';
         END IF;
\end{lstlisting}

d)

\begin{lstlisting}[language=sql]
  CREATE FUNCTION star (s CHAR(20)) RETURN INT
    IF(SELECT MIN(year) AS lowestYear, length
       FROM Movies
       WHERE length>120)
    RETURN s;
    ELSE RETURN 0;
    END IF;
\end{lstlisting}

e)

\begin{lstlisting}[language=sql]
  CREATE FUNCTION address (a CHAR(50)) RETURN CHAR
    IF (SELECT name
        FROM MovieStar
        WHERE address=a
        GROUP BY name
        HAVING COUNT(address)=1)
    RETURN a;
    ELSEIF SELECT name
           FROM MovieStar
           WHERE address=a
           GROUP BY name
           HAVING COUNT(address)=0 OR HAVING COUNT(address)>1
           RETURN NULL;
           END IF;
\end{lstlisting}

f)

\begin{lstlisting}[language=sql]
  CREATE FUNCTION nameStar (n CHAR(50)) RETURN CHAR
    (DELETE name FROM MovieStar WHERE n=name);
    UNION (DELETE title, movieTitle
          FROM StarsIn, Movies
          WHERE n=starName AND n=title);
\end{lstlisting}

\subsection*{Exercise 4.2}

a)

\begin{lstlisting}[language=sql]
  CREATE FUNCTION priceP (p INT) RETURN INT
    IF(SELECT MIN(price)
       FROM PC
       WHERE price=p); RETURN INT;
    END IF;
\end{lstlisting}

b)

\begin{lstlisting}[language=sql]
  CREATE FUNCTION productModel (m CHAR(20), n INT)
    IF(SELECT price
       FROM PC, Laptop, Printer
       WHERE m=maker AND n=model); RETURN price;
    END IF;
\end{lstlisting}

c)

\begin{lstlisting}[language=sql]
  CREATE FUNCTION takeALL (m INT, s INT, r INT, h INT, p INT ) RETURN CHAR
    INSERT INTO PC VALUES(m, s, r, h, p)
    IF(SELECT model, speed, ram, hd, price
       FROM PC
       WHERE model=m, speed=s, ram=r, hd=h, price=p); RETURN 'Error';
    ELSE model=model+1;
\end{lstlisting}

d)

\begin{lstlisting}[language=sql]
  CREATE FUNCTION morePrice (p INT) RETURN INT
    IF(SELECT price
       FROM Laptop, PC, Printer
       WHERE p=price
       HAVING COUNT(*)
       GROUP BY Laptop, PC, Printer;) RETURN INT;
    END IF;
\end{lstlisting}

\subsection*{Exercise 4.3}

a)

\begin{lstlisting}[language=sql]
  CREATE FUNCTION firepower (c CHAR(50)) RETURN INT
    IF(SELECT class, numGuns, bore
       FROM Classes WHERE c=class); RETURN numGuns*bore^3;
\end{lstlisting}

b)

\begin{lstlisting}[language=sql]
  CREATE FUNCTION battleName (n CHAR(20)) RETURN
    IF(SELECT name, country
       FROM Battles
       WHERE n=name
       HAVING COUNT (*)=2
       GROUP BY country); RETURN n;
    ELSEIF(SELECT name, country
           FROM Battles
           WHERE n=name
           HAVING COUNT(*)>2 OR HAVING COUNT(*)<2
           GROUP BY country; RETURN NULL;
\end{lstlisting}

c)

\begin{lstlisting}[language=sql]
  CREATE FUNCTION takeALL (c CHAR(20), t CHAR(20), c CHAR(20), n INT, b INT, d INT)
    INSERT INTO Classes VALUES(c, t, c, n, b, d);
    INSERT INTO Ships WHERE c=class;
\end{lstlisting}

d)

\begin{lstlisting}[language=sql]
  CREATE FUNCTION shipBattle (n CHAR(20))
    IF(SELECT date.B, launched.S
       FROM Battles.B, Ships.s
       WHERE date<launched;) RETURN date=0, launched=0;
    ENDIF;
\end{lstlisting}

\subsection*{Exercise 4.4}

\begin{align*}
  \sum_{i = 1}^{n}(x_{i} - \bar{x}^{2} / n)&=
    \frac{1}{n}\sum_{i = 1}^{n}(x_{i}^{2} - 2\bar{x}x_{i} + \bar{x}^{2}) \\
  &= \frac{1}{n}\left(\sum x_{i}^{2} - \sum 2\bar{x}x_{i} + \sum \bar{x}^{2}\right) \\
  &= \frac{1}{n}\left(\sum x_{i}^{2} - 2\bar{x}(n\bar{x}) + n\bar{x}^{2}\right) \\
  &= \frac{1}{n}\left(\sum x_{i}^{2} - n\bar{x}^{2}\right) \\
\end{align*}

\end{document}
